%
% File naaclhlt2012.tex
%

\documentclass[11pt,letterpaper]{article}
\usepackage{naaclhlt2012}
\usepackage{times}
\usepackage{latexsym}
\setlength\titlebox{6.5cm}    % Expanding the titlebox

\title{Statistical Machine Translation for Grammatical Error Correction}

\author{Arvind Srinivasan\\
	    Columbia University\\
	    {\tt vs2371@columbia.edu}
	  \And
	Louis Cialdella\\
  	Columbia University\\
  {\tt lmc2179@columbia.edu}}

\date{}

\begin{document}
\maketitle
\begin{abstract}
  This document contains the instructions for preparing a camera-ready
  manuscript for the proceedings of NAACL HLT 2012. The document itself conforms
  to its own specifications, and is therefore an example of what
  your manuscript should look like.  Authors are asked to conform to
  all the directions reported in this document.  Also included are instructions for
  submission to the conference, which are similar to the camera-ready instructions
  but remove author-identifying information.
\end{abstract}

\section{Background}

The following instructions are directed to authors of papers accepted
for publication in the NAACL HLT 2012 proceedings.  All authors are required
to adhere to these specifications. Authors are required to provide 
a Portable Document Format (PDF) version of
their papers.  The proceedings will be printed on US-Letter paper.
Authors from countries in which access to word-processing systems is
limited should contact the publication chairs as soon as possible.


\section{Task + Corpus}

This project was initiated as a submission to the ConLL 2013 Shared Task: Grammatical Error Correction \footnote{http://www.comp.nus.edu.sg/~nlp/conll13st.html}. As a result, all training and test data was from a provided corpus, the National University Singapore Corpus of Learner English, or NUCLE, and trained systems were scored with the provided NUS MaxMatch, or $M^2$ scorer. 

The corpus itself is comprised of ~1400 annotated essays written by english-second-language Singaporean students. The annotations provide an error type and correction, from the below types of errors:

\begin{tabular}{| l | c |}
  \hline
  tag & category \\
  \hline
  Vt & Verb tense \\
  Vm & Verb modal \\
  V0 & Missing verb \\ 
  Vform & Verb form \\
  SVA & Subject-verb-agreement \\
  ArtOrDet & Article or Determiner \\
  Nn & Noun number \\
  Npos & Noun possesive \\
  Pform & Pronoun form \\
  Pref & Pronoun reference \\
  Prep & Preposition \\
  Wci & Wrong collocation/idiom \\
  Wa & Acronyms \\
  Wform & Word form \\
  Wtone & Tone \\
  Srun & Runons, comma splice \\
  Smod & Dangling modifier \\
  Spar & Parallelism \\
  Sfrag & Fragment \\
  Ssub & Subordinate clause \\
  WOinc & Incorrect sentence form \\
  WOadv & Adverb/adjective position \\
  Trans & Link word/phrases \\
  Mec & Punctuation, capitalization, spelling, typos \\
  Rloc & Local redundancy \\
  Cit & Citation \\
  Others & Other errors \\
  Um & Unclear meaning (cannot be corrected) \\
  \hline
\end{tabular}

For the initial approach to this task, however, we looked to train a system that would identify a set of five errors: SVA, ArtOrDet, Nn, VForm, and Prep. The total corpus amounted to 67372 sentences, of which ~15,000 were noisy (references, urls, etc). Of that, 11288 sentences had annotations with those tags, leaving a fairly small dataset on which to train. After the time of this writing, ConLL has released an additional dataset - the blind test data with the gold references, so we anticipate that this dataset will grow significantly.

\section{Baseline}

Our baseline system used Moses to translate between a test corpus of original essays and the counterpart essays with all the annotations applied. This flattened corpus was split into ~58000 training sentences, ~2000 sentences for tuning, and ~8000 test sentences. 

Moses used Giza++ to align the sentences, with default heuristics and reordering models (grow-diag-final, msd-bidirectional-fe). The tuning used MERT with default features. 

To score the baseline, we use BLEU as well as the $M^2$ scorer. The $M^2$ scorer scores precision, accuracy, and F1 against the annotations, rather than the text itself. Though the conference version of $M^2$ is case-sensitive, we use a case-insensitive version for simplicity - recasing in English is a trivial task. Though both of these scorers support multi-reference evaluation, the corpus itself only has a single-reference gold reference file. Clearly, there are multiple equally fluent machine-generated correction candidates, even within the phrase table generated by the small training corpus, so we think this is an area that needs to be explored in terms of test corpus augmentation.

In reality, the optimality of a correction would be gauged by fluency and grammatical cohesion of the final generation, so neither Bleu nor $M^2$ adequately capture the ideal result - an ideal result would likely be modeled by a combination of the two, with multiple references. 


\subsection{Issues with Baseline}

\section{Corpus Cleaning + Datasets}

The first immediate issue with the baseline phrase table was that references and urls generated a large amount of noise in the correction data. Since we used the wordpunct NLTK tokenization scheme, urls in particular were inconsistently treated as multiple words, creating noisy alignments as well. Thus, the first cleaning step was to eliminate references completely, re-adding them after the correction. We accomplished this with a simple reegex substitution. 

Additionally, to experiment on the domain sensitivity of our system, we split the corpus by essay topic. This was particularly necessary because one of the challenges of the shared task was to submit system output for topics both inside and outside the training data. However, since the topics were not available a priori, we used an online version of the Latent Dirichlet Allocation (LDA) algorithm to generate a topic model, and then a K-Means clustering implementation to split the documents by those topics. We were then able to generate a held out dataset with 21572 training sentences (no references) with all but one topic, and 8695 test sentences (no references) with the remaining topic. This experiment showed that the NUCLE essay topics were largely similar in content, and that domain sensitivity is still a concern that ought to be tested by a heldout test set with a topic further removed from the training corpus.

To experiment with the impact of zero-annotation sentences in the training corpus, we also generated datasets without correct sentences. This left a dataset of 11288 sentences, of which we used 10000 to train and 288 to tune. 

To run stemming experiments, we used the stemmers bundled with NLTK. We stemmed and lemmatized the two corpora mentioned above with the Lancaster and Snowball stemmers and the WordNet lemmatizer to experiment with various degrees of stemming aggressiveness.

\subsection{Layout}
\label{ssec:layout}

Format manuscripts two columns to a page, in the manner these
instructions are formatted. The exact dimensions for a page on US-letter
paper are:

\begin{itemize}
\item Left and right margins: 1in
\item Top margin:1in
\item Bottom margin: 1in
\item Column width: 3.15in
\item Column height: 9in
\item Gap between columns: 0.2in
\end{itemize}

\noindent Papers should not be submitted on any other paper size. Exceptionally,
authors for whom it is \emph{impossible} to format on US-Letter paper,
may format for \emph{A4} paper. In this case, they should keep the \emph{top}
and \emph{left} margins as given above, use the same column width,
height and gap, and modify the bottom and right margins as necessary.
Note that the text will no longer be centered.

\subsection{The First Page}
\label{ssec:first}

Center the title, author's name(s) and affiliation(s) across both
columns (or, in the case of initial submission, space for the names). 
Do not use footnotes for affiliations.  Do not include the
paper ID number assigned during the submission process. 
Use the two-column format only when you begin the abstract.

{\bf Title}: Place the title centered at the top of the first page, in
a 15 point bold font.  (For a complete guide to font sizes and styles, see Table~\ref{font-table}.)
Long title should be typed on two lines without
a blank line intervening. Approximately, put the title at 1in from the
top of the page, followed by a blank line, then the author's names(s),
and the affiliation on the following line.  Do not use only initials
for given names (middle initials are allowed). Do not format surnames
in all capitals (e.g., ``Bangalore,'' not ``BANGALORE'').  The affiliation should
contain the author's complete address, and if possible an electronic
mail address. Leave about 0.75in between the affiliation and the body
of the first page.

{\bf Abstract}: Type the abstract at the beginning of the first
column.  The width of the abstract text should be smaller than the
width of the columns for the text in the body of the paper by about
0.25in on each side.  Center the word {\bf Abstract} in a 12 point
bold font above the body of the abstract. The abstract should be a
concise summary of the general thesis and conclusions of the paper.
It should be no longer than 200 words.  The abstract text should be in 10 point font.

{\bf Text}: Begin typing the main body of the text immediately after
the abstract, observing the two-column format as shown in 
the present document.  Do not include page numbers.

{\bf Indent} when starting a new paragraph. For reasons of uniformity,
use Adobe's {\bf Times Roman} fonts, with 11 points for text and 
subsection headings, 12 points for section headings and 15 points for
the title.  If Times Roman is unavailable, use {\bf Computer Modern
  Roman} (\LaTeX2e{}'s default; see section \ref{sect:pdf} above).
Note that the latter is about 10\% less dense than Adobe's Times Roman
font.

\subsection{Sections}

{\bf Headings}: Type and label section and subsection headings in the
style shown on the present document.  Use numbered sections (Arabic
numerals) in order to facilitate cross references. Number subsections
with the section number and the subsection number separated by a dot,
in Arabic numerals. 

{\bf Citations}: Citations within the text appear
in parentheses as~\cite{Gusfield:97} or, if the author's name appears in
the text itself, as Gusfield~\shortcite{Gusfield:97}. 
Append lowercase letters to the year in cases of ambiguities.  
Treat double authors as in~\cite{Aho:72}, but write as 
in~\cite{Chandra:81} when more than two authors are involved. 
Collapse multiple citations as in~\cite{Gusfield:97,Aho:72}.

\textbf{References}: Gather the full set of references together under
the heading {\bf References}; place the section before any Appendices,
unless they contain references. Arrange the references alphabetically
by first author, rather than by order of occurrence in the text.
Provide as complete a citation as possible, using a consistent format,
such as the one for {\em Computational Linguistics\/} or the one in the 
{\em Publication Manual of the American 
Psychological Association\/}~\cite{APA:83}.  Use of full names for
authors rather than initials is preferred.  A list of abbreviations
for common computer science journals can be found in the ACM 
{\em Computing Reviews\/}~\cite{ACM:83}.

The \LaTeX{} and Bib\TeX{} style files provided roughly fit the
American Psychological Association format, allowing regular citations, 
short citations and multiple citations as described above.

{\bf Appendices}: Appendices, if any, directly follow the text and the
references (but see above).  Letter them in sequence and provide an
informative title: {\bf Appendix A. Title of Appendix}.

\textbf{Acknowledgment} sections should go as a last (unnumbered) section immediately
before the references.  

\subsection{Footnotes}

{\bf Footnotes}: Put footnotes at the bottom of the page. They may
be numbered or referred to by asterisks or other
symbols.\footnote{This is how a footnote should appear.} Footnotes
should be separated from the text by a line.\footnote{Note the
line separating the footnotes from the text.}  Footnotes should be in 9 point font.

\subsection{Graphics}

{\bf Illustrations}: Place figures, tables, and photographs in the
paper near where they are first discussed, rather than at the end, if
possible.  Wide illustrations may run across both columns and should be placed at
the top of a page. Color illustrations are discouraged, unless you have verified that 
they will be understandable when printed in black ink.

\begin{table}
\begin{center}
\begin{tabular}{|l|rl|}
\hline \bf Type of Text & \bf Font Size & \bf Style \\ \hline
paper title & 15 pt & bold \\
author names & 12 pt & bold \\
author affiliation & 12 pt & \\
the word ``Abstract'' & 12 pt & bold \\
section titles & 12 pt & bold \\
document text & 11 pt  &\\
abstract text & 10 pt & \\
captions & 10 pt & \\
bibliography & 10 pt & \\
footnotes & 9 pt & \\
\hline
\end{tabular}
\end{center}
\caption{\label{font-table} Font guide. }
\end{table}

{\bf Captions}: Provide a caption for every illustration; number each one
sequentially in the form:  ``Figure 1. Caption of the Figure.'' ``Table 1.
Caption of the Table.''  Type the captions of the figures and 
tables below the body, using 10 point text.  

\section{Length of Submission}
\label{sec:length}

The NAACL HLT 2012 main conference accepts submissions of long papers
and short papers.  The maximum length of a long paper manuscript is
eight (8) pages of content and two (2) additional pages of references
\emph{only} (appendices count against the eight pages, not the
additional two pages).  The maximum length of a short paper manuscript
is four (4) pages including references.  For both long and short
papers, all illustrations, references, and appendices must be
accommodated within these page limits, observing the formatting
instructions given in the present document.  Papers that do not
conform to the specified length and formatting requirements are
subject to be rejected without review.

\section{Double-blind review process}
\label{sec:blind}

As the reviewing will be blind, the paper must not include the authors' names and affiliations. Furthermore, self-references that reveal the author's identity, e.g., "We previously showed (Smith, 1991) ..." must be avoided. Instead, use citations such as "Smith previously showed (Smith, 1991) ..." Papers that do not conform to these requirements will be rejected without review. In addition, please do not post your submissions on the web until after the review process is complete.

\section*{Acknowledgments}

Do not number the acknowledgment section.

\begin{thebibliography}{}

\bibitem[\protect\citename{Aho and Ullman}1972]{Aho:72}
Alfred~V. Aho and Jeffrey~D. Ullman.
\newblock 1972.
\newblock {\em The Theory of Parsing, Translation and Compiling}, volume~1.
\newblock Prentice-{Hall}, Englewood Cliffs, NJ.

\bibitem[\protect\citename{{American Psychological Association}}1983]{APA:83}
{American Psychological Association}.
\newblock 1983.
\newblock {\em Publications Manual}.
\newblock American Psychological Association, Washington, DC.

\bibitem[\protect\citename{{Association for Computing Machinery}}1983]{ACM:83}
{Association for Computing Machinery}.
\newblock 1983.
\newblock {\em Computing Reviews}, 24(11):503--512.

\bibitem[\protect\citename{Chandra \bgroup et al.\egroup }1981]{Chandra:81}
Ashok~K. Chandra, Dexter~C. Kozen, and Larry~J. Stockmeyer.
\newblock 1981.
\newblock Alternation.
\newblock {\em Journal of the Association for Computing Machinery},
  28(1):114--133.

\bibitem[\protect\citename{Gusfield}1997]{Gusfield:97}
Dan Gusfield.
\newblock 1997.
\newblock {\em Algorithms on Strings, Trees and Sequences}.
\newblock Cambridge University Press, Cambridge, UK.

\end{thebibliography}

\end{document}
