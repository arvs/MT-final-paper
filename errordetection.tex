%
% File naaclhlt2012.tex
%

\documentclass[11pt,letterpaper]{article}
\usepackage{naaclhlt2012}
\usepackage{times}
\usepackage{framed}
\usepackage{latexsym}
\setlength\titlebox{6.5cm}    % Expanding the titlebox

\title{Statistical Machine Translation for Grammatical Error Correction}

\author{Arvind Srinivasan\\
	    Columbia University\\
	    {\tt vs2371@columbia.edu}
	  \And
	Louis Cialdella\\
  	Columbia University\\
  {\tt lmc2179@columbia.edu}}

\date{}

\begin{document}
\maketitle
\begin{abstract}
  This document contains the instructions for preparing a camera-ready
  manuscript for the proceedings of NAACL HLT 2012. The document itself conforms
  to its own specifications, and is therefore an example of what
  your manuscript should look like.  Authors are asked to conform to
  all the directions reported in this document.  Also included are instructions for
  submission to the conference, which are similar to the camera-ready instructions
  but remove author-identifying information.
\end{abstract}

\section{Background}

This paper details our experiments in using Statistical Machine translation methods for 
error detection in English as part of the Conll-13 challenge. The challenge involved 
correction of errors made by Singaporean students of English

\section{Task + Corpus}

This project was initiated as a submission to the ConLL 2013 Shared Task: Grammatical Error Correction \footnote{http://www.comp.nus.edu.sg/~nlp/conll13st.html}. As a result, all training and test data was from a provided corpus, the National University Singapore Corpus of Learner English, or NUCLE, and trained systems were scored with the provided NUS MaxMatch, or $M^2$ scorer. 

The corpus itself is comprised of ~1400 annotated essays written by english-second-language Singaporean students. The annotations provide an error type and correction, from the below types of errors:

\begin{tabular}{| l | c |}
  \hline
  tag & category \\
  \hline
  Vt & Verb tense \\
  Vm & Verb modal \\
  V0 & Missing verb \\ 
  Vform & Verb form \\
  SVA & Subject-verb-agreement \\
  ArtOrDet & Article or Determiner \\
  Nn & Noun number \\
  Npos & Noun possesive \\
  Pform & Pronoun form \\
  Pref & Pronoun reference \\
  Prep & Preposition \\
  Wci & Wrong collocation/idiom \\
  Wa & Acronyms \\
  Wform & Word form \\
  Wtone & Tone \\
  Srun & Runons, comma splice \\
  Smod & Dangling modifier \\
  Spar & Parallelism \\
  Sfrag & Fragment \\
  Ssub & Subordinate clause \\
  WOinc & Incorrect sentence form \\
  WOadv & Adverb/adjective position \\
  Trans & Link word/phrases \\
  Mec & Punctuation, capitalization, spelling, typos \\
  Rloc & Local redundancy \\
  Cit & Citation \\
  Others & Other errors \\
  Um & Unclear meaning (cannot be corrected) \\
  \hline
\end{tabular}

For the initial approach to this task, however, we looked to train a system that would identify a set of five errors: SVA, ArtOrDet, Nn, VForm, and Prep. The total corpus amounted to 67372 sentences, of which ~15,000 were noisy (references, urls, etc). Of that, 11288 sentences had annotations with those tags, leaving a fairly small dataset on which to train. After the time of this writing, ConLL has released an additional dataset - the blind test data with the gold references, so we anticipate that this dataset will grow significantly.

\section{Problem Background: Types of error}
\indent We begin by motivating the approach we take by considering the problem in a little more depth. Two of the most common types of error are article errors and preposition errors. 
Article errors are most common for speakers of languages which themselves lack any sort of article 
(some very common examples are Chinese, Japanese, and Russian), though they are naturally present 
in the writings of most all students. Similarly, preposition errors account for a large number of 
errors, due to the vast complexity of the English preposition system. Preposition use is highly particular, 
and is often governed by the context in which the preposition appears (making it a good candidate for 
phrase based correction). \newline

\indent An additional major source of error is that of collocation (or idiomatic) errors. These 
arise when there is a strong association between words even though other choices might be semantically 
or syntactically correct (for example, strong computer makes sense, but powerful computer is idiomatically 
rendered). Additionally, this type of error encompasses stock idiomatic phrases where the individual words 
are not obviously related to the actual meaning (to “hit the nail on the head”, to “kick the bucket”, etc). 
These phrases are usually non-modifiable in many contexts or cannot be modified and retain their meaning 
unless great care is used. While rule-based approaches lead to extremely complex definitions, a 
statistical approach is well suited to collocation errors since it provides an easy and concrete way 
of taking into account the relationships between words and/or whole idiomatic phrases. \newline

\section{Baseline}
Our baseline system used Moses to "translate" between a test corpus of original essays and the counterpart essays with all the annotations applied. This "flattened" corpus was split into ~58000 training sentences, ~2000 sentences for tuning, and ~8000 test sentences. 

Moses used Giza++ to align the sentences, with default heuristics and reordering models (grow-diag-final, msd-bidirectional-fe). The tuning used MERT with default features. 

To score the baseline, we use BLEU as well as the $M^2$ scorer. The $M^2$ scorer scores precision, accuracy, and F1 against the annotations, rather than the text itself. Though the conference version of $M^2$ is case-sensitive, we use a case-insensitive version for simplicity - recasing in English is a trivial task. Though both of these scorers support multi-reference evaluation, the corpus itself only has a single-reference gold reference file. Clearly, there are multiple equally fluent machine-generated correction candidates, even within the phrase table generated by the small training corpus, so we think this is an area that needs to be explored in terms of test corpus augmentation.

In reality, the optimality of a "correction" would be gauged by fluency and grammatical cohesion of the final generation, so neither Bleu nor $M^2$ adequately capture the ideal result - an ideal result would likely be modeled by a combination of the two, with multiple references. 

\section{Issues with Baseline}
\indent The baseline displayed atrocious performance across the board. Two major and easily observable 
issues arise: the number of useless phrases in the phrase table, and the bias of phrases towards not 
making changes. \newline
\indent Firstly, we note that the vast majority of phrases in the phrase table are entirely unhelpful 
in decoding. For example, the number of possible translations of "(" is obscenely large, having hundreds
of entries. Of these, only a handful are useful, and even then the most common substitution should still 
be "(", as no errors using the left paren character occur anyway. This happens for virtually every piece 
of punctuation and trivial phrase, and does nothing except slow down decoding and cause spurious translation 
issues during decoding. We deal with a solution to this in the phrase table pruning section. \newline
\indent A second issue is that the phrase table entries are biased towards not making changes when 
decoding happens. That is, for a given phrase, the phrase table causes there to be a much larger chance 
of said phrase remaining unchanged in the final product, rather than changing and potentially correcting 
an error. A fairly typical PT entry indicative of this issue is: \newline
\newline
\indent , according to ||| , according to the ||| 0.166667 0.99065 0.0384615 0.578006 2.718 ||| 0-0 1-1 2-2 ||| 6 26 1 \newline \newline
\indent , according to ||| , according to ||| 0.961538 0.99065 0.961538 0.986281 2.718 ||| 0-0 1-1 2-2 ||| 26 26 25 \newline \newline
\indent We note here that the bottom phrase (which makes no changes) has a probability of 
0.961538, while the top phrase has a probability of 0.0384615. We talk about our approach to this 
in the section on downsampling. \newline
\indent Additionally, we run into the problem of the BLEU metric, which causes problems with both 
tuning and evaluation. In this task, the BLEU metric used by Moses is quite a poor indicator 
of success. BLEU only checks for particular N-Grams in the reference, and in this task there is only a 
single reference sentence. This means that BLEU measures only N-grammatic distance to the supplied reference, 
which makes no guarantee of fluency. That is, a given change might drastically improve a sentence and make an 
otherwise incorrect sentence mostly or completely correct, but this may not be reflected in the BLEU score if 
it does not completely line up with the reference sentence. This leads to both strange tuning artifacts (such 
as the over-tuning described above) and difficult-to-interpret evaluation results. \newline
\section{Corpus cleaning and Datasets}

The first immediate issue with the baseline phrase table was that references and urls generated a large amount of noise in the correction data. Since we used the wordpunct NLTK tokenization scheme, urls in particular were inconsistently treated as multiple words, creating noisy alignments as well. Thus, the first cleaning step was to eliminate references completely, re-adding them after the correction. We accomplished this with a simple reegex substitution. 

Additionally, to experiment on the domain sensitivity of our system, we split the corpus by essay topic. This was particularly necessary because one of the challenges of the shared task was to submit system output for topics both inside and outside the training data. However, since the topics were not available a priori, we used an online version of the Latent Dirichlet Allocation (LDA) algorithm to generate a topic model, and then a K-Means clustering implementation to split the documents by those topics. We were then able to generate a "held out" dataset with 21572 training sentences (no references) with all but one topic, and 8695 test sentences (no references) with the remaining topic. This experiment showed that the NUCLE essay topics were largely similar in content, and that domain sensitivity is still a concern that ought to be tested by a heldout test set with a topic further removed from the training corpus.

To experiment with the impact of zero-annotation sentences in the training corpus, we also generated datasets without correct sentences. This left a dataset of 11288 sentences, of which we used 10000 to train and 288 to tune. 

To run stemming experiments, we used the stemmers bundled with NLTK~\cite{nltk}. We stemmed and lemmatized the two corpora mentioned above with the Lancaster and Snowball stemmers and the WordNet lemmatizer to experiment with various degrees of stemming aggressiveness.

We used a Moses (SRILM) ngram language model trained on the test and dev datasets. Given the topic closeness between the test and train sets and the relatively small size of the test set, there was not a significant OOV problem, and experiments with big language models trained on the Brown Corpus did not yield any benefits to either of the scoring metrics.

\section{Approach}
\subsection{Significance Testing}
\subsection{Downsampling}
\subsection{Stemming}


One of the issues with longer phrase choice, especially with verb and SVA errors, is that of sparsit. In general, these are context determined errors, so the ideal scenario would have a phrase table with a single source phrase translating to a target phrase, rather than multiple candidates that represent the different error type. 

Take, for example, the phrase "The dog had walked." With a verb or SVA error, this can be formulated as "The dog had walks", "The dog had walk", "The dog had walking," etc. However, there is only one valid annotation. In this case, it makes sense to reduce the word to its base form on the source side, because there will then be 4 training examples rather than one per type. In this trivial example, this also intuitively makes decoding less ambiguous, as reduction to base forms 

Though this could be achieved with segmentation as well (and that could be a different experiment), we decided that stemming was more likely to yield a consistent representation of this theory. We experimented with various levels of stemming aggressiveness (the WordNet lemmatizer on the low end, and the Lancaster stemmer on the high end), using our different training sets to measure the impact.

\subsection{NoCorrect - Only Errors}

The primary issue with the baseline system was the bias towards a correct $\rightarrow$ correct translation in decoding. We tried to address this in various ways, but the naive way to do this was to train a system that dropped all sentences without annotations altogether, essentially forcing the SMT system to learn the annotation types with high probability. A variant of this system would experiment with various "factors" of correctness in the training data, where each 

We ran two experiments on this. The first was to test the performance on this system on a test set that only had sentences with annotations. The second was to test it on the full test set that we used for the other experiments. The former was to establish the hypothesis that decoding performance was dependent on a linear relationship between the number of "errors" in the training sets and the test sets, and the latter was to stay consistent with the methodology of the other experiments. In practice, this method is infeasible - without another model to recognize correctness, there is no way to decode "only sentences with errors" in a blind test, or production setting.


\section{Results}

All the experiments we ran achieved consistently high Bleu scores, but this includes the sentences that were already correct and mapped to another correct sentence. When the test set was scored with BLEU without any modifications, it achieved a BLEU score of of 87.32, 95.8/90.6/85.5/80.6. \\

\begin{tabular}{ |l|l|}
\hline
\textbf{Experiment} & \textbf{Bleu} & \textbf{} & \textbf{} \\ \hline
\textit{Baseline} & \textit{96.34, 98.7/97.1/95.6/94.1}\\ \hline
\textit{Lancaster} & \textit{73.04, 88.2/77.6/68.6/60.6} \\ \hline
\textit{WordNet} & \textit{86.48, 94.1/88.8/84.1/79.6} \\ \hline
\textit{HeldOut} & \textit{96.81, 98.9/97.5/96.2/94.9} \\ \hline
\textit{NoCorrect} & \textit{96.46, 98.7/97.2/95.7/94.3} \\ \hline
\textit{Stem+NoCorr} & \textit{88.26, 95.8/90.8/86.0/81.6} \\ \hline
\textit{HeldOut+Sig} & \textit{96.80, 98.9/97.5/96.1/94.8} \\ \hline
\textit{NoCorr+Sig} & \textit{95.59, 98.5/96.5/94.6/92.8 } \\ \hline
\textit{Lanc+Sig} & \textit{89.01, 96.3/91.2/87.0/82.9} \\ \hline
\textit{Word+Sig} & \textit{92.87, 97.4/94.4/91.6/88.8} \\ \hline
\end{tabular}
\\\\

With the $M^2$ scorer, the different experiments yielded marginal improvements. In this case, the starting point for the score was 0.00, since the precision/recall/f1 were evaluated against only the generated and gold annotations. \\

\begin{tabular}{ |l|l|l|l|l|l| }
\hline
\textit{$M^2$} & \textit{Precision} & Recall & F1 \\ \hline
\textit{Baseline} & \textit{0} & 0 & 0 \\ \hline
\textit{Lancaster} & \textit{0.0215} & 0.1915 & 0.0387 \\ \hline
\textit{WordNet} & \textit{0.049} & 0.1773 & 0.0768 \\ \hline
\textit{HeldOut} & \textit{0.0116} & 0.0663 & 0.0198 \\ \hline
\textit{NoCorrect} & \textit{0.0838} & 0.2555 & 0.1262 \\ \hline
\textit{Stem+NoCorr} & \textit{0.0624} & 0.3086 & 0.1038 \\ \hline
\textit{HeldOut+Sig} & \textit{0.0086} & 0.0312 & 0.0134 \\ \hline
\textit{NoCorr+Sig} & \textit{0.0268} & 0.1031 & 0.0425 \\ \hline
\textit{Lanc+Sig} & \textit{0.0073} & 0.0312 & 0.0118 \\ \hline
\textit{Word+Sig} & \textit{0.0221} & 0.0906 & 0.0355 \\ \hline
\end{tabular}
\\\\

For the significance testing experiments, a significant amount of the phrase table was pruned of statistically insignificant entries. \\

\begin{tabular}{|l|l|}
\hline
\textit{M2} & \textit{\% of PT pruned} \\ \hline
\textit{HeldOut+Sig} & \textit{56.90\%} \\ \hline
\textit{NoCorr+Sig} & \textit{53.17\%} \\ \hline
\textit{Lanc+Sig} & \textit{51.81\%} \\ \hline
\textit{Word+Sig} & \textit{57.57\%} \\ \hline
\end{tabular}

\section{Observations}

\subsection{Significance Testing}
\subsection{Downsampling}
\subsection{Stemming}

The results from the stemming experiments were a mixed bag, and probably suffered from the fact that it was a broad-stroke approach when a finer brush might have been preferable. Take, for example, the following phrase table entries for the Lancaster experiment:

\begin{framed}
wait $\|\|$ to wait $\|\|$ 0.333333 1 0.047619 0.00710397 2.718 $\|\|$ 0-1 $\|\|$ 3 21 1 \\
wait $\|\|$ wait $\|\|$ 1 1 0.428571 0.45 2.718 $\|\|$ 0-0 $\|\|$ 9 21 9 \\
wait $\|\|$ waiting $\|\|$ 1 1 0.52381 0.55 2.718 $\|\|$ 0-0 $\|\|$ 11 21 11
\end{framed}

\begin{framed}
' s fin $\|\|$ 's final $\|\|$ 1 0.25 0.0769231 0.15583 2.718 $\|\|$ 0-0 1-0 2-1 $\|\|$ 1 13 1 \\
' s fin $\|\|$ 's finances $\|\|$ 1 0.25 0.384615 0.0249327 2.718 $\|\|$ 0-0 1-0 2-1 $\|\|$ 5 13 5 \\
' s fin $\|\|$ 's financial $\|\|$ 1 0.25 0.538462
\end{framed}

These two phrase table entries are indicative of a few characteristic trends of the stemming experiments:
\begin{enumerate}
  \item The stemming had the intended effect on the verb entries in the phrase table, appropriately recognizing different verb forms with roughly equal probability. The first example shows the rather interesting characteristic that the verb "wait" is rarely corrected to the infinitive form, which, without stemming, would not have been found, as there would have been a single entry with "to wait" $\rightarrow$ "to wait" unless there was a specific annotation that added the infinitive.

  \item The stemming had the opposite effect on nouns and adjectives, introducing ambiguity where there was previously none. In the second example, stemming to "fin" meant that there were three candidates for a single base form, where only one was really valid in the context of another sentence. 

  \item An additional insight was that tokenized punctuation can either be helpful or specifically unhelpful in the context of error correction. In the second example, the \{' s\} serves as potentially useful disambiguation - the following word can only be a noun, adjective, or adverb, since it is a possessive particle. However, parentheses, ellipses, and commas are unhelpful phrase boundaries, and should probably have been dropped altogether. Though we did not consider the punctuation error type for the initial tests, we anticipate that this would have been problematic had we done so.
  
\end{enumerate}

Some of the additional noise introduced with stemming was fixed by being less aggressive (specifically the noise characteristic of the latter example), and in our tests, the WordNet lemmatizer performed best. In future tests (see sec. 10), we want to isolate error types in decoding sets to solidify the claim that stemming specifically increases recall and precision on Verb form and SVA errors, and does not help in the other cases - the alternative is that stemming merely introduces ambiguity in the same way downsampling does, and so it achieved slight gains just due to overcorrection.

\subsection{NoCorrect - Only Errors}

Though not included in the above table, the test of our NoCorrect system only decoding sentences with errors had the most encouraging results, as shown below:

\begin{framed}
BLEU = 88.26, 95.8/90.8/86.0/81.6 \\
$M^2$ Precision: 0.4697 \\
$M^2$ Recall   : 0.1951 \\
$M^2$ F1       : 0.2757 \\
\end{framed}

What this tells us is that at least for these 5 error forms, training a system on just errors is the best approach - it intuitively overcorrects in the same way that stemming or downsampling does artificially. The challenge then becomes twofold:
\begin{enumerate}
  \item Increase the size of the training set significantly - after dropping incorrect sentences and noise, the training set was only 10000 sentences.
  \item Drop sentences that are probably correct before decoding -- as the results show, the performance significantly drops with the full training set.
\end{enumerate}

Additionally, it remains to be seen how this approach generalizes to the other error types. The five error types that we dealt with are largely single word corrections, so the SMT system was at least able to learn the most common word substitutions. For longer error types, like fragments, runons, redundancy, and incorrect sentence form, the NoCorrect system would probably have much more damaging overcorrection issues.

\section{Conclusions}

Our results did not do much to validate SMT as a robust approach for error correction, as our experiments more or less failed across the board. Although these were only preliminary experiments, there were a few error types that seemed to have systemic issues when addressed in an SMT framework, and a few of the experiments (NoCorrect and Significance Testing in particular) were entirely designed to mitigate some of those systemic issues rather than introduce any particular linguistic nuance to the correction task. That said, part of this was self inflicted - the errors that we focused on for the shared task were those that may have been best served by a parse-tree rule based approach. The SMT approach, on the other hand, would have been able to find results for error types that we did not address due to sparsity of annotation, but are not tractable in a rule framework, like fragment correction or idiomatic consistency. We also avoided some tricky reordering issues that may have resulted from error types like redundancy, or anything that depended on related clauses. 

That said, there is a significant body of SMT research that introduces syntactic information into the translation task, and SMT is clearly an important approach to pursue for error correction, even only because rules are highly language dependent, suffer significantly in the presence of noise, and do not account for the inherent ambiguity in the error correction task. We hope that our research guides other researchers to more interesting conclusions in the future.

\section{Future Work}

\section*{Acknowledgments}

We especially want to thank Professors Habash and Tomeh for their guidance and patience during this project. We both learned a lot, and are excited to continue investigating approaches to SMT and error correction in the future.

\bibliography{nltk}

\begin{thebibliography}{}

\bibitem[\protect\citename{Aho and Ullman}1972]{Aho:72}
Alfred~V. Aho and Jeffrey~D. Ullman.
\newblock 1972.
\newblock {\em The Theory of Parsing, Translation and Compiling}, volume~1.
\newblock Prentice-{Hall}, Englewood Cliffs, NJ.

\bibitem[\protect\citename{{American Psychological Association}}1983]{APA:83}
{American Psychological Association}.
\newblock 1983.
\newblock {\em Publications Manual}.
\newblock American Psychological Association, Washington, DC.

\bibitem[\protect\citename{{Association for Computing Machinery}}1983]{ACM:83}
{Association for Computing Machinery}.
\newblock 1983.
\newblock {\em Computing Reviews}, 24(11):503--512.

\bibitem[\protect\citename{Chandra \bgroup et al.\egroup }1981]{Chandra:81}
Ashok~K. Chandra, Dexter~C. Kozen, and Larry~J. Stockmeyer.
\newblock 1981.
\newblock Alternation.
\newblock {\em Journal of the Association for Computing Machinery},
  28(1):114--133.

\bibitem[\protect\citename{Gusfield}1997]{Gusfield:97}
Dan Gusfield.
\newblock 1997.
\newblock {\em Algorithms on Strings, Trees and Sequences}.
\newblock Cambridge University Press, Cambridge, UK.

\end{thebibliography}

\end{document}
